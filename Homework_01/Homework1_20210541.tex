
\documentclass[12pt, letterpaper]{amsart}
\usepackage[left=1in,right=1in,bottom=0.5in,top=0.8in]{geometry}
\usepackage{amsfonts}
\usepackage{amsmath, amssymb}
\usepackage{graphicx}
\usepackage[font=small,labelfont=bf]{caption}
\usepackage{epstopdf}
\usepackage[pdfpagelabels,hyperindex]{hyperref}
\usepackage{xcolor}
\usepackage{amsthm}
\usepackage{float}
\usepackage{pgfplots}
\usepackage{listings}
\usepackage{tikz-cd}
\usepackage{longtable}
\usepackage{mathrsfs}
\usepackage{mathtools}
\usepackage{hyperref}
\usepackage[most]{tcolorbox}
\usepackage{amssymb}
\usepackage{algorithm}
\usepackage{algpseudocode}
\usepackage{comment}
\usepackage{mathpartir}


\linespread{1.2} 

\hypersetup{
pdftitle={.},
pdfauthor={Hojune Lee},
}

\newcommand{\overbar}[1]{\mkern 1.5mu\overline{\mkern-1.5mu#1\mkern-1.5mu}\mkern 1.5mu}
\DeclareRobustCommand{\stirling}{\genfrac\{\}{0pt}{}}
\newtheorem{thm}{Theorem}[subsection]
\newtheorem{cor}[thm]{Corollary}
\newtheorem{prop}[thm]{Proposition}
\newtheorem{lem}[thm]{Lemma}
\newtheorem{conj}[thm]{Conjecture}
\newtheorem{quest}[thm]{Question}
\newtheorem{ppty}[thm]{Property}
\newtheorem{ppties}[thm]{Properties}
\newtheorem{axiom}[thm]{Axiom}
\newtheorem{claim}[thm]{Claim}
\newtheorem{prob}[thm]{Problem}


\theoremstyle{definition}
\newtheorem{defn}[thm]{Definition}
\newtheorem{defns}[thm]{Definitions}
\newtheorem{con}[thm]{Construction}
\newtheorem{exmp}[thm]{Example}
\newtheorem{exmps}[thm]{Examples}
\newtheorem{notn}[thm]{Notation}
\newtheorem{notns}[thm]{Notations}
\newtheorem{addm}[thm]{Addendum}
\newtheorem{exer}[thm]{Exercise}
\newtheorem{limit}[thm]{Limitation}


\theoremstyle{remark}
\newtheorem{rem}[thm]{Remark}
\newtheorem{rems}[thm]{Remarks}
\newtheorem{warn}[thm]{Warning}
\newtheorem{sch}[thm]{Scholium}


% newcommand bb
    \newcommand{\BA}{{\mathbb {A}}} \newcommand{\BB}{{\mathbb {B}}}
    \newcommand{\BC}{{\mathbb {C}}} \newcommand{\BD}{{\mathbb {D}}}
    \newcommand{\BE}{{\mathbb {E}}} \newcommand{\BF}{{\mathbb {F}}}
    \newcommand{\BG}{{\mathbb {G}}} \newcommand{\BH}{{\mathbb {H}}}
    \newcommand{\BI}{{\mathbb {I}}} \newcommand{\BJ}{{\mathbb {J}}}
    \newcommand{\BK}{{\mathbb {U}}} \newcommand{\BL}{{\mathbb {L}}}
    \newcommand{\BM}{{\mathbb {M}}} \newcommand{\BN}{{\mathbb {N}}}
    \newcommand{\BO}{{\mathbb {O}}} \newcommand{\BP}{{\mathbb {P}}}
    \newcommand{\BQ}{{\mathbb {Q}}} \newcommand{\BR}{{\mathbb {R}}}
    \newcommand{\BS}{{\mathbb {S}}} \newcommand{\BT}{{\mathbb {T}}}
    \newcommand{\BU}{{\mathbb {U}}} \newcommand{\BV}{{\mathbb {V}}}
    \newcommand{\BW}{{\mathbb {W}}} \newcommand{\BX}{{\mathbb {X}}}
    \newcommand{\BY}{{\mathbb {Y}}} \newcommand{\BZ}{{\mathbb {Z}}}

% newcommand  scr
    \newcommand{\sA}{{\mathscr {A}}} \newcommand{\sB}{{\mathscr {B}}}
    \newcommand{\sC}{{\mathscr {C}}} \newcommand{\sD}{{\mathscr {D}}}
    \newcommand{\sE}{{\mathscr {E}}} \newcommand{\sF}{{\mathscr {F}}}
    \newcommand{\sG}{{\mathscr {G}}} \newcommand{\sH}{{\mathscr {H}}}
    \newcommand{\sI}{{\mathscr {I}}} \newcommand{\sJ}{{\mathscr {J}}}
    \newcommand{\sK}{{\mathscr {K}}} \newcommand{\sL}{{\mathscr {L}}}
    \newcommand{\sN}{{\mathscr {N}}} \newcommand{\sM}{{\mathscr {M}}}
    \newcommand{\sO}{{\mathscr {O}}} \newcommand{\sP}{{\mathscr {P}}}
    \newcommand{\sQ}{{\mathscr {Q}}} \newcommand{\sR}{{\mathscr {R}}}
    \newcommand{\sS}{{\mathscr {S}}} \newcommand{\sT}{{\mathscr {T}}}
    \newcommand{\sU}{{\mathscr {U}}} \newcommand{\sV}{{\mathscr {V}}}
    \newcommand{\sW}{{\mathscr {W}}} \newcommand{\sX}{{\mathscr {X}}}
    \newcommand{\sY}{{\mathscr {Y}}} \newcommand{\sZ}{{\mathscr {Z}}}


% newcommand cal
    \newcommand{\CA}{{\mathcal {A}}} \newcommand{\CB}{{\mathcal {B}}}
    \newcommand{\CC}{{\mathcal {C}}} \newcommand{\CD}{{\mathcal {D}}}
    \newcommand{\CE}{{\mathcal {E}}} \newcommand{\CF}{{\mathcal {F}}}
    \newcommand{\CG}{{\mathcal {G}}} \newcommand{\CH}{{\mathcal {H}}}
    \newcommand{\CI}{{\mathcal {I}}} \newcommand{\CJ}{{\mathcal {J}}}
    \newcommand{\CK}{{\mathcal {K}}} \newcommand{\CL}{{\mathcal {L}}}
    \newcommand{\CM}{{\mathcal {M}}} \newcommand{\CN}{{\mathcal {N}}}
    \newcommand{\CO}{{\mathcal {O}}} \newcommand{\CP}{{\mathcal {P}}}
    \newcommand{\CQ}{{\mathcal {Q}}} \newcommand{\CR}{{\mathcal {R}}}
    \newcommand{\CS}{{\mathcal {S}}} \newcommand{\CT}{{\mathcal {T}}}
    \newcommand{\CU}{{\mathcal {U}}} \newcommand{\CV}{{\mathcal {V}}}
    \newcommand{\CW}{{\mathcal {W}}} \newcommand{\CX}{{\mathcal {X}}}
    \newcommand{\CY}{{\mathcal {Y}}} \newcommand{\CZ}{{\mathcal {Z}}}

    % newcommand frak
     \newcommand{\fa}{{\mathfrak{a}}}  \newcommand{\fb}{{\mathfrak{b}}}
     \newcommand{\fc}{{\mathfrak{c}}}  \newcommand{\fd}{{\mathfrak{d}}}
     \newcommand{\fe}{{\mathfrak{e}}}  \newcommand{\ff}{{\mathfrak{f}}}
     \newcommand{\fg}{{\mathfrak{g}}}  \newcommand{\fh}{{\mathfrak{h}}}
     \newcommand{\fii}{{\mathfrak{i}}}  \newcommand{\fj}{{\mathfrak{j}}}
     \newcommand{\fk}{{\mathfrak{m}}}  \newcommand{\fl}{{\mathfrak{l}}}
     \newcommand{\fm}{{\mathfrak{m}}}  \newcommand{\fn}{{\mathfrak{n}}}
     \newcommand{\fo}{{\mathfrak{o}}}  \newcommand{\fp}{{\mathfrak{p}}}
     \newcommand{\fq}{{\mathfrak{q}}}  \newcommand{\fr}{{\mathfrak{r}}}
     \newcommand{\fs}{{\mathfrak{s}}}  \newcommand{\ft}{{\mathfrak{t}}}
     \newcommand{\fu}{{\mathfrak{u}}}  \newcommand{\fv}{{\mathfrak{v}}}
     \newcommand{\fw}{{\mathfrak{w}}}  \newcommand{\fx}{{\mathfrak{x}}}
     \newcommand{\fy}{{\mathfrak{y}}}  \newcommand{\fz}{{\mathfrak{z}}}

    \newcommand{\fA}{{\mathfrak{A}}}  \newcommand{\fB}{{\mathfrak{B}}}
     \newcommand{\fC}{{\mathfrak{C}}}  \newcommand{\fD}{{\mathfrak{D}}}
     \newcommand{\fE}{{\mathfrak{E}}}  \newcommand{\fF}{{\mathfrak{F}}}
     \newcommand{\fG}{{\mathfrak{G}}}  \newcommand{\fH}{{\mathfrak{H}}}
     \newcommand{\fI}{{\mathfrak{I}}}  \newcommand{\fJ}{{\mathfrak{J}}}
     \newcommand{\fK}{{\mathfrak{K}}}  \newcommand{\fL}{{\mathfrak{L}}}
     \newcommand{\fM}{{\mathfrak{M}}}  \newcommand{\fN}{{\mathfrak{N}}}
     \newcommand{\fO}{{\mathfrak{O}}}  \newcommand{\fP}{{\mathfrak{P}}}
     \newcommand{\fQ}{{\mathfrak{Q}}}  \newcommand{\fR}{{\mathfrak{R}}}
     \newcommand{\fS}{{\mathfrak{S}}}  \newcommand{\fT}{{\mathfrak{T}}}
     \newcommand{\fU}{{\mathfrak{U}}}  \newcommand{\fV}{{\mathfrak{V}}}
     \newcommand{\fW}{{\mathfrak{W}}}  \newcommand{\fX}{{\mathfrak{X}}}
     \newcommand{\fY}{{\mathfrak{Y}}}  \newcommand{\fZ}{{\mathfrak{Z}}}



 % newcommand :rm
     \newcommand{\RA}{{\mathrm {A}}} \newcommand{\RB}{{\mathrm {B}}}
    \newcommand{\RC}{{\mathrm {C}}} \newcommand{\RD}{{\mathrm {D}}}
    \newcommand{\RE}{{\mathrm {E}}} \newcommand{\RF}{{\mathrm {F}}}
    \newcommand{\RG}{{\mathrm {G}}} \newcommand{\RH}{{\mathrm {H}}}
    \newcommand{\RI}{{\mathrm {I}}} \newcommand{\RJ}{{\mathrm {J}}}
    \newcommand{\RK}{{\mathrm {K}}} \newcommand{\RL}{{\mathrm {L}}}
    \newcommand{\RM}{{\mathrm {M}}} \newcommand{\RN}{{\mathrm {N}}}
    \newcommand{\RO}{{\mathrm {O}}} \newcommand{\RP}{{\mathrm {P}}}
    \newcommand{\RQ}{{\mathrm {Q}}} \newcommand{\RR}{{\mathrm {R}}}
    \newcommand{\RS}{{\mathrm {S}}} \newcommand{\RT}{{\mathrm {T}}}
    \newcommand{\RU}{{\mathrm {U}}} \newcommand{\RV}{{\mathrm {V}}}
    \newcommand{\RW}{{\mathrm {W}}} \newcommand{\RX}{{\mathrm {X}}}
    \newcommand{\RY}{{\mathrm {Y}}} \newcommand{\RZ}{{\mathrm {Z}}}

    \newcommand{\Ad}{{\mathrm{Ad}}} \newcommand{\Aut}{{\mathrm{Aut}}}
    \newcommand{\Br}{{\mathrm{Br}}} \newcommand{\Ch}{{\mathrm{Ch}}}
    \newcommand{\cod}{{\mathrm{cod}}} \newcommand{\cont}{{\mathrm{cont}}}
    \newcommand{\cl}{{\mathrm{cl}}}   \newcommand{\Cl}{{\mathrm{Cl}}}
    \newcommand{\disc}{{\mathrm{disc}}}\newcommand{\Eis}{{\mathrm{Eis}}}
    \newcommand{\Div}{{\mathrm{Div}}} \renewcommand{\div}{{\mathrm{div}}}
    \newcommand{\End}{{\mathrm{End}}} \newcommand{\Frob}{{\mathrm{Frob}}}
    \newcommand{\Gal}{{\mathrm{Gal}}} \newcommand{\GL}{{\mathrm{GL}}}
    \newcommand{\Hom}{{\mathrm{Hom}}} \renewcommand{\Im}{{\mathrm{Im}}}
    \newcommand{\Ind}{{\mathrm{Ind}}} \newcommand{\ind}{{\mathrm{ind}}}
    \newcommand{\inv}{{\mathrm{inv}}}
    \newcommand{\Isom}{{\mathrm{Isom}}} \newcommand{\Jac}{{\mathrm{Jac}}}
    \newcommand{\ad}{{\mathrm{ad}}}  \newcommand{\Tr}{{\mathrm{Tr}}}
    \newcommand{\Ker}{{\mathrm{Ker}}} \newcommand{\Ros}{{\mathrm{Ros}}}
    \newcommand{\Lie}{{\mathrm{Lie}}} \newcommand{\Hol}{{\mathrm{Hol}}}

    \newcommand{\cyc}{{\mathrm{cyc}}}\newcommand{\id}{{\mathrm{id}}}
    \newcommand{\new}{{\mathrm{new}}} \newcommand{\NS}{{\mathrm{NS}}}
    \newcommand{\ord}{{\mathrm{ord}}} \newcommand{\rank}{{\mathrm{rank}}}
    \newcommand{\PGL}{{\mathrm{PGL}}} \newcommand{\Pic}{\mathrm{Pic}}
    \newcommand{\cond}{\mathrm{cond}} \newcommand{\Is}{{\mathrm{Is}}}
    \renewcommand{\Re}{{\mathrm{Re}}} \newcommand{\reg}{{\mathrm{reg}}}
    \newcommand{\Res}{{\mathrm{Res}}} \newcommand{\Sel}{{\mathrm{Sel}}}
    \newcommand{\RTr}{{\mathrm{Tr}}} \newcommand{\alg}{{\mathrm{alg}}}
    \newcommand{\PSL}{{\mathrm{PSL}}}

\newcommand{\coker}{{\mathrm{coker}}}
\newcommand{\val}{{\mathrm{val}}} \newcommand{\sign}{{\mathrm{sign}}}
\newcommand{\mult}{{\mathrm{mult}}} \newcommand{\Vol}{{\mathrm{Vol}}}
\newcommand{\Meas}{{\mathrm{Meas}}}\renewcommand{\mod}{\ \mathrm{mod}\ }
\newcommand{\Ann}{\mathrm{Ann}}
\newcommand{\Tor}{\mathrm{Tor}}
\newcommand{\Supp}{\mathrm{Supp}}\newcommand{\supp}{\mathrm{supp}}
\newcommand{\Max}{\mathrm{Max}}
\newcommand{\Coker}{\mathrm{Coker}}
\newcommand{\Stab}{\mathrm{Stab}}
\newcommand{\Irr}{\mathrm{Irr}}\newcommand{\Inf}{\mathrm{Inf}}\newcommand{\Sup}{\mathrm{Sup}}
\newcommand{\rk}{\mathrm{rk}}\newcommand{\Fil}{\mathrm{Fil}}
\newcommand{\Sim}{{\mathrm{Sim}}} \newcommand{\SL}{{\mathrm{SL}}}
\newcommand{\Spec}{{\mathrm{Spec}}} \newcommand{\SO}{{\mathrm{SO}}}
\newcommand{\SU}{{\mathrm{SU}}} \newcommand{\Sym}{{\mathrm{Sym}}}
\newcommand{\sgn}{{\mathrm{sgn}}} \newcommand{\tr}{{\mathrm{tr}}}
\newcommand{\tor}{{\mathrm{tor}}}  \newcommand{\ur}{{\mathrm{ur}}}
\newcommand{\vol}{{\mathrm{vol}}}  \newcommand{\ab}{{\mathrm{ab}}}
\newcommand{\Sh}{{\mathrm{Sh}}} \newcommand{\Ell}{{\mathrm{Ell}}}
\newcommand{\Char}{{\mathrm{Char}}}\newcommand{\Tate}{{\mathrm{Tate}}}
\newcommand{\corank}{{\mathrm{corank}}} \newcommand{\Cond}{{\mathrm{Cond}}}
\newcommand{\Inn}{{\mathrm{Inn}}} \newcommand{\Spf}{{\mathrm{Spf}}}
\newcommand{\Mat}{{\mathrm{Mat}}}


    \font\cyr=wncyr10  \newcommand{\Sha}{\hbox{\cyr X}}
    \newcommand{\wt}{\widetilde} \newcommand{\wh}{\widehat} \newcommand{\ck}{\check}
    \newcommand{\pp}{\frac{\partial\bar\partial}{\pi i}}
    \newcommand{\pair}[1]{\langle {#1} \rangle}
    \newcommand{\wpair}[1]{\left\{{#1}\right\}}
    \newcommand{\intn}[1]{\left( {#1} \right)}
    \newcommand{\norm}[1]{\|{#1}\|}
    \newcommand{\sfrac}[2]{\left( \frac {#1}{#2}\right)}
    \newcommand{\ds}{\displaystyle}
    \newcommand{\ov}{\overline}
    \newcommand{\Gros}{Gr\"{o}ssencharaktere}
    \newcommand{\incl}{\hookrightarrow}
    \newcommand{\lra}{\longrightarrow}
     \newcommand{\ra}{\rightarrow}
    \newcommand{\imp}{\Longrightarrow}
    \newcommand{\lto}{\longmapsto}
    \newcommand{\bs}{\backslash}
    \newcommand{\nequiv}{\equiv\hspace{-7.8pt}/}
    \theoremstyle{plain}


\definecolor{energy}{RGB}{114,0,172}
\definecolor{freq}{RGB}{45,177,93}
\definecolor{spin}{RGB}{251,0,29}
\definecolor{signal}{RGB}{203,23,206}
\definecolor{circle}{RGB}{217,86,16}
\definecolor{average}{RGB}{203,23,206}
\newcommand{\K}{\operatornamewithlimits{K}}
\colorlet{shadecolor}{gray!20}
\pgfplotsset{compat=1.9}
\def\N{10}
\def\M{4}
\usepgflibrary{fpu}


\def\upint{\mathchoice%
    {\mkern13mu\overline{\vphantom{\intop}\mkern7mu}\mkern-20mu}%
    {\mkern7mu\overline{\vphantom{\intop}\mkern7mu}\mkern-14mu}%
    {\mkern7mu\overline{\vphantom{\intop}\mkern7mu}\mkern-14mu}%
    {\mkern7mu\overline{\vphantom{\intop}\mkern7mu}\mkern-14mu}%
  \int}
\def\lowint{\mkern3mu\underline{\vphantom{\intop}\mkern7mu}\mkern-10mu\int}



\makeatletter
\let\c@equation\c@thm
\raggedbottom
\makeatother
\numberwithin{equation}{section}
%--------Meta Data: Fill in your info------
\author[Hojune Lee]{Hojune Lee,\ \ 20210541}

\title{HW01}
\begin{document}

\maketitle


\textbf{Exercise 2.15.} Verify that $\lambda^{-1}(-)$ and its $\beta, \eta$-rules do infact imply our original elimination $\beta, \eta$-rule. 

\vspace{4mm}

\begin{proof}
    
Let's make clear understanding about this question first. 
First, $\lambda^{-1}(-)$ is direct representation of forward isomorphism between $Tm(\Gamma, \Pi(A, B)) \cong Tm(\Gamma.A, B)$, which means 
\[\iota_{\Gamma, A, B} : Tm(\Gamma, \Pi(A, B)) \xrightarrow{\cong} Tm(\Gamma.A, B)\]
\\
which means that, it is a kind of homomorphism that for given $f : \Pi(A, B)$ , construct term $\lambda^{-1}(f)$ in context $\Gamma.A$ which type is $B$. 
By the given rule, we write 
\[
\inferrule
{
    \Gamma \vdash f : \Pi(A, B)
}
{
    \Gamma.A \vdash \lambda^{-1}(f) : B 
}
\]
\\
In this case, defining $\beta, \eta$-rule is quite simple, each are 
\begin{mathpar}
\inferrule
{
    \Gamma \vdash A \text{ type} \quad \Gamma.A \vdash b : B 
}
{
    \Gamma.A \vdash \lambda^{-1}(\lambda(b)) = b : B 
}
\and  
\inferrule 
{
    \Gamma \vdash A \text{ type} \quad \Gamma.A \vdash B \text{ type} \quad \Gamma \vdash f : \Pi(A, B)
}
{
    \Gamma \vdash \lambda(\lambda^{-1}(f)) = f : \Pi(A, B)
}
\end{mathpar}
\\
Here, if there exists a term $a$ such that $\Gamma \vdash a : A$, then we can construct an 
substitution $\Gamma \vdash \mathbf{id}.a : \Gamma.A$ and then we can pull-back above $\lambda^{-1}(f)$ through $\mathbf{id}.a$. Then we can write, 
\[
\inferrule
{
    \Gamma \vdash f : \Pi(A, B) \quad \Gamma \vdash a : A 
}
{
    \Gamma \vdash \lambda^{-1}(f)[\mathbf{id}.a] : B[\mathbf{id}.a]
}
\]
\\
The result term is what we defined originally as $\mathbf{app}(f, a) := \lambda^{-1}(f)[\mathbf{id}.a]$. So, the question is that 
can we derive our original $\beta, \eta$-rules for $\mathbf{app}(-, -)$ via above rules? 
Here, such original $\beta, \eta$-rule is written by 
\begin{mathpar}
\inferrule
{
    \Gamma \vdash a : A \quad \Gamma.A \vdash b : B 
}
{
    \Gamma \vdash \mathbf{app}(\lambda(b), a) = b[\mathbf{id}.a] : B[\mathbf{id}.a] 
}
\and  
\inferrule 
{
    \Gamma \vdash A \text{ type} \quad \vdash \Gamma.A \vdash B \text{ type} \quad \Gamma \vdash f : \Pi(A, B)
}
{
    \Gamma \vdash \lambda(\mathbf{app}(f[\mathbf{p}], \mathbf{q})) = f : \Pi(A, B)
}
\end{mathpar}
\\
Let's prove first one. Since we have $\Gamma \vdash A \text{ type}$ as pre-supposition, 
we get  
\[\inferrule{\Gamma \vdash A \text{ type} \quad \Gamma.A \vdash b : B}{\Gamma.A \vdash \lambda^{-1}(\lambda(b)) = b : B}\]
\\
However, since $\Gamma \vdash a : A$ is given, we can construct substitution 
\[\Gamma \vdash \mathbf{id}.a : \Gamma.A\]
So we can pull-back the result into context $\Gamma$, 
\[\Gamma \vdash \lambda^{-1}(\lambda(b))[\mathbf{id}.a] = b[\mathbf{id}.a] : B[\mathbf{id}.a]\]
\\
Recap the definition of $\mathbf{app}(-, -)$. We can rewrite above result by 
\[\Gamma \vdash \mathbf{app}(\lambda(b), a) = b[\mathbf{id}.a] : B[\mathbf{id}.a]\]
\\
This proved the first one. Let's prove second one. 
Before prove this, we must argue the naturality of forward homomorphism $\iota_{\Gamma, A, B}$ in our version. 
Suppose that following diagram, 
\[
\begin{tikzcd}[row sep = huge, column sep = huge]
Tm(\Gamma, \Pi(A, B)) \arrow[d, "\gamma^*"] \arrow[r, "\iota_{\Gamma}"] & Tm(\Gamma.A, B)  \arrow[d, "\gamma.A^*"]\\
Tm(\Delta, \Pi(A[\gamma], B[\gamma.A])) \arrow[r, "\iota_{\Delta}"] & Tm(\Delta.A[\gamma], B[\gamma.A])
\end{tikzcd}
\]
\\
Here, $\gamma^*$ and $\gamma.A^*$ notate that works of pull-back operator w.r.t. $\gamma$ and $\gamma.A$. 
And since we've construct such isomorphisms via $\lambda^{-1}$, to satisfy naturality of $\iota$ here, following must holds : 
\[
\inferrule
{
    \Delta \vdash \gamma : \Gamma \quad \Gamma \vdash A \text{ type} \quad \Gamma.A \vdash B \text{ type} \quad \Gamma \vdash f : \Pi(A, B) 
}
{
    \Delta.A[\gamma] \vdash \lambda^{-1}(f[\gamma]) = \lambda^{-1}(f)[\gamma.A] : B[\gamma.A]
}
\]
\\
This can be easily obtained via following application of above diagram. 
\[
\begin{tikzcd}[row sep = huge, column sep = huge]
    \Gamma \vdash f : \Pi(A, B) \arrow[r, "\cong \text{  by  } \lambda^{-1}"] \arrow[d, "\text{pull-back by  } \gamma"] & \Gamma.A \vdash \lambda^{-1}(f) : B \arrow[d, blue, "\text{pull-back by  } \gamma.A"]\\
    \Delta \vdash f[\gamma] : \Pi(A[\gamma], B[\gamma.A]) \arrow[red, r, "\cong \text{  by  } \lambda^{-1}"] & \Delta.A[\gamma] \vdash \textcolor{red}{\lambda^{-1}(f[\gamma])} = \textcolor{blue}{\lambda^{-1}(f)[\gamma.A]} : B[\gamma.A]
\end{tikzcd}
\]
We can use above rule for proving second one by :
\[
\inferrule
{
    \Gamma.A \vdash \mathbf{p} : \Gamma \quad \Gamma \vdash A \text{ type} \quad \Gamma.A \vdash B \text{ type} \quad \Gamma \vdash f : \Pi(A, B)
}
{
    \Gamma.A.A[\mathbf{p}] \vdash \lambda^{-1}(f[\mathbf{p}]) = \lambda^{-1}(f)[\mathbf{p}.A] : B[\mathbf{p}.A]
}
\]
\\
However, since there exists a term $\Gamma.A \vdash \mathbf{q} : A[\mathbf{p}]$, we can construct well-formed substitution $\Gamma.A \vdash \mathbf{id}.\mathbf{q} : \Gamma.A.A[\mathbf{p}]$. 
Let's pull-back above result by $\mathbf{id}.\mathbf{q}$ : 
\[\Gamma.A \vdash \lambda^{-1}(f[\mathbf{p}])[\mathbf{id}.\mathbf{q}] = \lambda^{-1}(f)[\mathbf{p}.A][\mathbf{id}.\mathbf{q}] : B[\mathbf{p}.A][\mathbf{id}.\mathbf{q}]\]
\\
However, we already know that $(\mathbf{p}.A) \circ (\mathbf{id}.\mathbf{q}) = \mathbf{id}$, rewrite above by : 
\[\Gamma.A \vdash \lambda^{-1}(f[\mathbf{p}])[\mathbf{id}.\mathbf{q}] = \lambda^{-1}(f) : B\]
\\
Here, by introduction rule, 
\[\Gamma \vdash \lambda(\lambda^{-1}(f[\mathbf{p}])[\mathbf{id}.\mathbf{q}]) = \lambda(\lambda^{-1}(f)) : \Pi(A, B)\]
\\
Now almost done. Rewrite $\lambda^{-1}(f[\mathbf{p}][\mathbf{id}.\mathbf{q}])$ as $\mathbf{app}(f[\mathbf{p}], \mathbf{q})$ and rewrite $\lambda(\lambda^{-1}(f))$ as $f$. 
\[\Gamma \vdash \lambda(\mathbf{app}(f[\mathbf{p}], \mathbf{q})) = f : \Pi(A, B)\]
\\
This is end of proof. 
\end{proof}

\newpage

\textbf{Exercise 2.16.} Show that using $\Sigma$ types we can define a non-dependent pair type 
whose formation rule states that if $\Gamma \vdash A \text{ type}$ and $\Gamma \vdash B \text{ type}$ then 
$\Gamma \vdash A \times B \text{ type}$. Then define the introduction and 
elimination rules from section 2.1 for this encoding, and check that $\beta, \eta$-rules holds. 

\begin{proof}
    
Our goal is that construct following rules of non-dependent pair type can be defined via dependent sum type $\Sigma$. 
\begin{mathpar}
\inferrule
{
    \Gamma \vdash A  \text{ type} \quad \Gamma \vdash B  \text{ type}
}
{
    \Gamma \vdash A \times B \text{ type}
}
\and  
\inferrule
{
    \Gamma \vdash a : A \quad \Gamma \vdash b : B 
}
{
    \Gamma \vdash (a, b) : A \times B
}
\and  
\inferrule
{
    \Gamma \vdash p : A \times B 
}
{
    \Gamma \vdash \mathbf{fst}(p) : A 
}
\and
\inferrule
{
    \Gamma \vdash p : A \times B 
}
{
    \Gamma \vdash \mathbf{snd}(p) : B
}
\end{mathpar}
\begin{mathpar}
\inferrule
{
    \Gamma \vdash a : A \quad \Gamma \vdash b : B 
}
{
    \Gamma \vdash \mathbf{fst}((a, b)) = a : A 
}
\and
\inferrule
{
    \Gamma \vdash a : A \quad \Gamma \vdash b : B 
}
{
    \Gamma \vdash \mathbf{snd}((a, b)) = b : B 
}
\and  
\inferrule
{
    \Gamma \vdash p : A \times B 
}
{
    \Gamma \vdash p = (\mathbf{fst}(p), \mathbf{snd}(p)) : A \times B 
}
\end{mathpar}

\begin{mathpar}
\inferrule
{
    \Delta \vdash \gamma : \Gamma \quad \Gamma \vdash A, B \text{ type} 
}
{
    \Delta \vdash (A \times B)[\gamma] = A[\gamma] \times B[\gamma] \text{ type}
}
\and 
\inferrule
{
    \Delta \vdash \gamma : \Gamma \quad \Gamma \vdash a : A, b : B \quad \Gamma \vdash (a, b) : A \times B 
}
{
    \Delta \vdash (a, b)[\gamma] = (a[\gamma], b[\gamma]) : A[\gamma] \times B[\gamma] 
}
\end{mathpar}
\\
First, let's define $A \times B$ first. Since $\Gamma \vdash A \text{ type}$ implies that $\Gamma.A \vdash B[\mathbf{p}] \text{ type}$, so we can define 
\[A \times B := \Sigma(A, B[\mathbf{p}])\] 
\\
Then remainings can be shown easily. 
If $\Gamma \vdash a : A$ and $\Gamma \vdash b : B$, then $\Gamma \vdash b : B[\mathbf{id}] = B[\mathbf{p}][\mathbf{id}.a]$, so 
\[
\inferrule
{
    \Gamma \vdash a : A \quad \Gamma \vdash b : B
}
{ \Gamma \vdash (a, b) := \mathbf{pair}(a, b) : \Sigma(A, B[\mathbf{p}]) }\]
\\
$\mathbf{fst}, \mathbf{snd}$ can be defined by : 
\[
\inferrule
{
    \Gamma \vdash p : \Sigma(A, B[\mathbf{p}])
}
{
    \Gamma \vdash \mathbf{fst}(p) : A  \qquad \Gamma \vdash \mathbf{snd}(p) : B[\mathbf{p}][\mathbf{id}.\mathbf{fst}(p)] = B
}
\]
\\
Then we can also see that such construction holds naturality. 
\begin{mathpar}
\inferrule
{
    \Delta \vdash \gamma : \Gamma \quad \Gamma \vdash A, B \text{ type}
}
{
    \Delta \vdash (A \times B)[\gamma] = \Sigma(A, B[\mathbf{p}])[\gamma] = \Sigma(A[\gamma], B[\mathbf{p}][\gamma.A]) = \Sigma(A[\gamma], B[\gamma][\mathbf{p}]) = A[\gamma] \times B[\gamma] \text{ type}
}
\end{mathpar}
\\
\[
    \inferrule
    {
        \Delta \vdash \gamma : \Gamma \quad \Gamma \vdash a : A, b : B \quad \Gamma \vdash (a, b) : A \times B 
    }
    {
        \Delta \vdash (a, b)[\gamma] = \mathbf{pair}(a, b)[\gamma] = \mathbf{pair}(a[\gamma], b[\gamma]) = (a[\gamma], b[\gamma]) : A[\gamma] \times B[\gamma]
    }
\]
\\
Here, the type-former works like 
\[\bigtimes_{\Gamma} : Ty(\Gamma) \times Ty(\Gamma) \rightarrow Ty(\Gamma) \]
\\
And the term natural isomorphism is constructed via 
\[\iota : Tm(\Gamma, A) \times Tm(\Gamma, B) \cong Tm(\Gamma, A \times B)\]

Remaining is that check for the $\beta, \eta$-rules. 

\[
\inferrule
{
    \Gamma \vdash a : A \quad \Gamma.A \vdash B[\mathbf{p}] \text{ type} \quad \Gamma \vdash b : B = B[\mathbf{p}][\mathbf{id}.a]
}
{
    \Gamma \vdash \mathbf{fst}((a, b)) = \mathbf{fst}(\mathbf{pair}(a,  b)) = a : A
}
\]
\[
\inferrule
{
    \Gamma \vdash a : A \quad \Gamma.A \vdash B[\mathbf{p}] \text{ type} \quad \Gamma \vdash b : B = B[\mathbf{p}][\mathbf{id}.a]
}
{
    \Gamma \vdash \mathbf{snd}((a, b)) = \mathbf{snd}(\mathbf{pair}(a,  b)) = b : B
}
\]
Finally, 
\[
\inferrule
{
    \Gamma \vdash A \text{ type} \quad \Gamma.A \vdash B[\mathbf{p}] \text{ type} \quad \Gamma \vdash p : A \times B = \Sigma(A, B[\mathbf{p}])
}
{
    \Gamma \vdash p = \mathbf{pair}(\mathbf{fst}(p), \mathbf{snd}(p)) = (\mathbf{fst}(p), \mathbf{snd}(p)) : A \times B 
}
\]
\\ 
So our definition $A \times B := \Sigma(A, B[\mathbf{p}])$ holds on whole requirements. 

\end{proof}

\newpage 

\textbf{Exercise 2.24.} Fixing $\Delta \vdash \gamma : \Gamma$, prove that there is at most one substitution $\Delta \vdash \overline{\gamma} : \Gamma.\mathbf{Void}$ such that satisfying $\mathbf{p} \circ \overline{\gamma} = \gamma$. 
\begin{proof}
I'll draw diagram for good understanding. 
\[
\begin{tikzcd}
    \Delta \arrow[r, "\gamma"] \arrow[rd, "\overline{\gamma}"']& \Gamma \arrow[d, "?"', shift right]
    \\ & \Gamma.\mathbf{Void} \arrow[u, "p"', shift right]
\end{tikzcd}
\]
We'll use following rule for substitution : 
\[
\inferrule
{
    \Gamma \vdash A \text{ type} \quad \Delta \vdash \gamma : \Gamma.A 
}
{
    \Delta \vdash \gamma = (\mathbf{p} \circ \gamma).\mathbf{q}[\gamma] : \Gamma.A
}
\tag{*}
\]
Suppose that $\overline{\gamma}_i, i = 1, 2$ such that $\Delta \vdash \overline{\gamma_i} : \Gamma.\mathbf{Void}, i = 1,2$ and $(\mathbf{p} \circ \overline{\gamma}_i) = \gamma, i = 1, 2$ exists. 
Then we can apply (*) by 
\[
\inferrule
{
    \Gamma \vdash \mathbf{Void} \text{ type} \quad \Delta \vdash \overline{\gamma}_i : \Gamma.\mathbf{Void}
}
{
    \Delta \vdash \overline{\gamma}_i = (\mathbf{p} \circ \overline{\gamma}_i).\mathbf{q}[\overline{\gamma}_i] : \Gamma.\mathbf{Void}
}
\]
By assumption, we can directly rewrite above result by 
\[\Delta \vdash \overline{\gamma}_i = \gamma.\mathbf{q}[\overline{\gamma}_i] : \Gamma.\mathbf{Void}\tag{**}\]
\\
However, $\Gamma.\mathbf{Void} \vdash \mathbf{q} : \mathbf{Void}$. So, we can pull-back $\mathbf{q}$ through $\overline{\gamma}_i$. 
\[
\inferrule
{
    \Delta \vdash \overline{\gamma}_i : \Gamma.\mathbf{Void} \quad \Gamma.\mathbf{Void} \vdash \mathbf{q} : \mathbf{Void}
}
{
    \Delta \vdash \mathbf{q}[\overline{\gamma}_i] : \mathbf{Void}[\overline{\gamma}_i] = \mathbf{Void}
}
\]
However, the result of Exercise 2.23. implies that, 
\[\Delta \vdash \mathbf{q}[\overline{\gamma}_1] = \mathbf{q}[\overline{\gamma}_2] : \mathbf{Void}\]
\\
We can use this result for (**). 
\[\Delta \vdash \overline{\gamma}_1 = \gamma.\mathbf{q}[\overline{\gamma}_1] = \gamma.\mathbf{q}[\overline{\gamma}_2] = \overline{\gamma}_2 : \Gamma.\mathbf{Void}\]
This implies that, 
\[\Delta \vdash \overline{\gamma}_1 = \overline{\gamma}_2 : \Gamma.\mathbf{Void}\]
We proved that, if such $\overline{\gamma}$ exists, then it is unique. This is sufficient to prove that 
there is at most one such $\overline{\gamma}$. 

\end{proof}

\newpage 

\textbf{Exercise 2.25.} Let $\Gamma.\mathbf{Void} \vdash A \text{ type}$ and $\Gamma \vdash a : A[\mathbf{id}.b]$. Show that 
$\Gamma.\mathbf{Void} \vdash A[(\mathbf{id}.b) \circ \mathbf{p}] = A \text{ type}$, and therefore that $\Gamma.\mathbf{Void} \vdash a[\mathbf{p}] : A$. 

\begin{proof}
    
We'll use following previous result. 
\[(\gamma.a) \circ \delta = (\gamma \circ \delta).a[\delta]\]
\\
In this exercies, we can write as 
\[(\mathbf{id}.b) \circ \mathbf{p} = (\mathbf{id} \circ \mathbf{p}).b[\mathbf{p}]\]
\\
However, by property of $\mathbf{id}$, 
\[(\mathbf{id}.b) \circ \mathbf{p} = (\mathbf{id} \circ \mathbf{p}).b[\mathbf{p}] = \mathbf{p}.b[\mathbf{p}]\tag{*}\] \\
However, our problem gives some pre-supposition : 
\[\Gamma \vdash A[\mathbf{id}.b] \text{ type}\] \\
Which means that 
\[\Gamma \vdash \mathbf{id}.b : \Gamma.\mathbf{Void}\] 
This also gives pre-supposition : 
\[\Gamma \vdash b : \mathbf{Void}\] 
So we can pull-back such $b$ through $\mathbf{p}$ : 
\[\Gamma.\mathbf{Void} \vdash b[\mathbf{p}] : \mathbf{Void}[\mathbf{p}] = \mathbf{Void}\] \\
However, $\Gamma.\mathbf{Void} \vdash \mathbf{q} : \mathbf{Void}$ and Exercise 2.23. implies that 
\[\Gamma.\mathbf{Void} \vdash b[\mathbf{p}] = \mathbf{q} : \mathbf{Void}\] \\
Then we can rewrite (*) by 
\[\]\[(\mathbf{id}.b) \circ \mathbf{p} = (\mathbf{id} \circ \mathbf{p}).b[\mathbf{p}] = \mathbf{p}.b[\mathbf{p}] = \mathbf{p}.\mathbf{q} = \mathbf{id}\] \\
in context $\Gamma.\mathbf{Void}$. So we can prove our original claim, 
\[\Gamma.\mathbf{Void} \vdash A[(\mathbf{id}.b) \circ \mathbf{p}] = A[\mathbf{id}] = A \text{ type}\] \\
Then the remaining part is easy. Since $\Gamma \vdash a : A[\mathbf{id}.b]$, we can pull-back $a$ into $\Gamma.\mathbf{Void}$ through $p$ : 
\[\Gamma.\mathbf{Void} \vdash a[\mathbf{p}] : A[\mathbf{id}.b][\mathbf{p}] = A[(\mathbf{id}.b)\circ \mathbf{p}] = A\] \\
\end{proof}

\newpage 

\textbf{Exercise 2.26.} Derive the following rule, using the previous exercise and the $\eta$-rule. 
\[
\inferrule
{
    \Gamma \vdash b : \mathbf{Void} \quad \Gamma.\mathbf{Void} \vdash A \text{ type} \quad \Gamma \vdash a : A[\mathbf{id}.b]
}
{
    \Gamma \vdash a = \mathbf{absurd}(b) : A[\mathbf{id}.b]
}
\] 

\begin{proof}

We'll use above Exercise 2.25. as lemma. We proved that 
\[
\inferrule
{
    \Gamma \vdash b : \mathbf{Void} \quad  \Gamma.\mathbf{Void} \vdash A \quad \text{ type} \quad \Gamma \vdash a : A[\mathbf{id}.b]
}
{
    \Gamma.\mathbf{Void} \vdash a[\mathbf{p}] : A
}
\tag{*}
\] \\
And the original $\eta$-rule is that 
\[
\inferrule
{
    \Gamma \vdash b : \mathbf{Void} \quad \Gamma.\mathbf{Void} \vdash a : A 
}
{
    \Gamma \vdash \mathbf{absurd}(b) = a[\mathbf{id}.b] : A[\mathbf{id}.b]
}
\] \\
To apply this, let's write that 
\[
\inferrule
{
    \Gamma \vdash b : \mathbf{Void} \quad \Gamma.\mathbf{Void} \vdash a[\mathbf{p}] : A 
}
{
    \Gamma \vdash \mathbf{absurd}(b) = a[\mathbf{p}][\mathbf{id}.b] : A[\mathbf{id}.b]
}
\]\\
Since (*) gives the premises, we get 
\[\Gamma \vdash \mathbf{absurd}(b) = a[\mathbf{p}][\mathbf{id}.b] : A[\mathbf{id}.b]\]
\\
However, $a[\mathbf{p}][\mathbf{id}.b] = a[\mathbf{p}\circ (\mathbf{id}.b)] = a[\mathbf{id}] = a$, we can get 
\[\Gamma \vdash  \mathbf{absurd}(b) = a : A[\mathbf{id}.b]\]
\\
This is End of Proof. 
\end{proof}

\newpage 

\textbf{Exercise 2.29.} Give rules axiomatizing the boolean analogue of $\mathbf{absurd}'$, and prove 
that these rules are interderivable with our rules for $\mathbf{if}(a_t, a_f, b)$. 

\begin{proof}
    
This is also question about \lq cut-rule applied' version of our backward homomorphism $\iota^{-1}$. 
Since 
\[\iota^{-1} : Tm(\Gamma, A[\mathbf{id}.\mathbf{true}]) \times Tm(\Gamma, A[\mathbf{id}.\mathbf{false}]) \rightarrow Tm(\Gamma.\mathbf{Bool}, A) \]
\\
Then when we directly construct this homomorphism, it will be 
\[
\inferrule
{
    \Gamma.\mathbf{Bool} \vdash A \text{ type} \quad \Gamma \vdash a_t : A[\mathbf{id}.\mathbf{true}] \quad \Gamma \vdash a_f : A[\mathbf{id}.\mathbf{false}]
}
{
    \Gamma.\mathbf{Bool} \vdash \mathbf{if}'(a_t, a_f) : A 
}
\]
\\
With above definition, the remaining modified rules are simply constructed : 
\[
\inferrule
{
    \Delta \vdash \gamma : \Gamma \quad \Gamma.\mathbf{Bool} : A \text{ type} \quad \Gamma \vdash a_t : A[\mathbf{id}.\mathbf{true}] \quad \Gamma \vdash a_f : A[\mathbf{id}.\mathbf{false}]
}
{
    \Delta.\mathbf{Bool} \vdash \mathbf{if}'(a_t, a_f)[\gamma.\mathbf{Bool}] = \mathbf{if}'(a_t[\gamma], a_f[\gamma]) : A[\gamma.\mathbf{Bool}]
}
\]
\\
\[
\inferrule
{
    \Gamma.\mathbf{Bool} \vdash A \text{ type} \quad \Gamma \vdash a_t : A[\mathbf{id}.\mathbf{true}] \quad \Gamma \vdash a_f : A[\mathbf{id}.\mathbf{false}]
}
{
    \Gamma \vdash \mathbf{if}'(a_t, a_f)[\mathbf{id}.\mathbf{true}] = a_t : A[\mathbf{id}.\mathbf{true}]
}
\]
\\
\[
\inferrule
{
    \Gamma.\mathbf{Bool} \vdash A \text{ type} \quad \Gamma \vdash a_t : A[\mathbf{id}.\mathbf{true}] \quad \Gamma \vdash a_f : A[\mathbf{id}.\mathbf{false}]
}
{
    \Gamma \vdash \mathbf{if}'(a_t, a_f)[\mathbf{id}.\mathbf{false}] = a_f : A[\mathbf{id}.\mathbf{false}]
}
\]
\\
\[
\inferrule
{
    \Gamma.\mathbf{Bool} \vdash A \text{ type} \quad \Gamma.\mathbf{Bool} \vdash a : A
}
{
    \Gamma.\mathbf{Bool} \vdash \mathbf{if}'(a[\mathbf{id}.\mathbf{true}], a[\mathbf{id}.\mathbf{false}]) = a : A}
\]
\\
However, our rules for $\mathbf{if}$ is constructed on context $\Gamma$. Actually, 
as same we discussed, it can be imagined as pull-backed $\mathbf{if}'$ into $\Gamma$. Here, we can write 
\[
\inferrule
{
    \Gamma.\mathbf{Bool} \vdash A \text{ type} \quad \Gamma \vdash a_t : A[\mathbf{id}.\mathbf{true}] \quad \Gamma \vdash a_f : A[\mathbf{id}.\mathbf{false}] \quad \Gamma \vdash b : \mathbf{Bool}
}
{
    \Gamma \vdash \mathbf{if}'(a_t, a_f)[\mathbf{id}.b] := \mathbf{if}(a_t, a_f, b) : A[\mathbf{id}.b]
}
\]
\\
Now, let's prove our original rules for $\mathbf{if}$ implies above $\beta, \eta$-rules. ( Here, I'll omit some premises when they are clear )
We have 
\[\Gamma \vdash \mathbf{if}(a_t, a_f, \mathbf{true}) = a_t : A[{\mathbf{id.true}}]\]
Rewrite this : 
\[\Gamma \vdash \mathbf{if'}(a_t, a_f)[\mathbf{id.true}] = a_t : A[\mathbf{id.true}]\]
Similarly, 
\[\Gamma \vdash \mathbf{if'}(a_t, a_f)[\mathbf{id.false}] = a_f : A[\mathbf{id.false}]\]
\\
This directly proves above $\beta$-rules. For $\eta$-rule, we can imagine following diagram : 

\[
\begin{tikzcd}
    \Gamma & \Gamma.\mathbf{Bool} \arrow[l, "p"]  \arrow[shift left, r, "id.q"] &  \Gamma.\mathbf{Bool.Bool} \arrow[l, "p.\mathbf{Bool}"]
\end{tikzcd}
\]

How can we use above intuition? Let's see. The original $\eta$-rule was 
\[
\inferrule
{\Gamma.\mathbf{Bool} \vdash A \text{ type} \quad \Gamma.\mathbf{Bool} \vdash a : A \quad \Gamma \vdash b : \mathbf{Bool} }
{ \Gamma \vdash \mathbf{if}(a[\mathbf{id.true}], a[\mathbf{id.false}], b) = a[\mathbf{id}.b] : A[\mathbf{id}.b]}
\]
\\
To prove the $\eta$-rule for $\mathbf{if'}$ of us, we'll use that 
\[
\inferrule
{
    \Delta \vdash \gamma : \Gamma \quad \Gamma.\mathbf{Bool} : A \text{ type} \quad \Gamma \vdash a_t : A[\mathbf{id}.\mathbf{true}] \quad \Gamma \vdash a_f : A[\mathbf{id}.\mathbf{false}]
}
{
    \Delta.\mathbf{Bool} \vdash \mathbf{if}'(a_t, a_f)[\gamma.\mathbf{Bool}] = \mathbf{if}'(a_t[\gamma], a_f[\gamma]) : A[\gamma.\mathbf{Bool}]
}
\]
\\
We can write this as 
\[
\inferrule
{
    \Gamma.\mathbf{Bool} \vdash \mathbf{p} : \Gamma \quad \Gamma.\mathbf{Bool} \vdash a : A \quad \Gamma \vdash a[\mathbf{id.true}] : A[\mathbf{id.true}] \quad \Gamma \vdash a[\mathbf{id.false}] : A[\mathbf{id.false}]
}
{
    \Gamma.\mathbf{Bool}.\mathbf{Bool} \vdash \mathbf{if'}(a[\mathbf{id.true}], a[\mathbf{id.false}])[\mathbf{p.Bool}] = \mathbf{if'}(a[\mathbf{id.true}][\mathbf{p}], a[\mathbf{id.false}][\mathbf{p}]) : A[\mathbf{p.Bool}]
}
\]
We can pull-back the result using $\Gamma.\mathbf{Bool} \vdash \mathbf{q} : \mathbf{Bool}$. 
\[\Gamma.\mathbf{Bool} \vdash \mathbf{if'}(a[\mathbf{id.true}], a[\mathbf{id.false}])[\mathbf{p.Bool}][\mathbf{id.q}] \] \[= \mathbf{if'}(a[\mathbf{id.true}][\mathbf{p}], a[\mathbf{id.false}][\mathbf{p}])[\mathbf{id.q}] : A[\mathbf{p.Bool}][\mathbf{id.q}]\]
\\
Rewrite this as 
\[
\Gamma.\mathbf{Bool} \vdash \mathbf{if'}(a[\mathbf{id.true}], a[\mathbf{id.false}]) = \mathbf{if}(a[\mathbf{id.true}][\mathbf{p}], a[\mathbf{id.false}][\mathbf{p}], \mathbf{q}) : A 
\]
\\
However, we already know that $\mathbf{id.true} \circ \mathbf{p} = \mathbf{p}.A \circ \mathbf{id.true'}$ where $\mathbf{id.true}$ is arrow $\Gamma.\mathbf{Bool} \rightarrow \Gamma.\mathbf{Bool.Bool}$. Then ,  
\[
\Gamma.\mathbf{Bool} \vdash \mathbf{if'}(a[\mathbf{id.true}], a[\mathbf{id.false}]) = \mathbf{if}(a[\mathbf{p}.A][\mathbf{id.true'}], a[\mathbf{p}.A][\mathbf{id.false'}], \mathbf{q}) : A 
\]
However, the $\eta$-rule of given $\mathbf{if}$ on $\Gamma.\mathbf{Bool}$ context implies that 
\[\mathbf{if'}(a[\mathbf{id.true}], a[\mathbf{id.false}]) = \mathbf{if}(a[\mathbf{p}.A][\mathbf{id.true'}], a[\mathbf{p}.A][\mathbf{id.false'}], \mathbf{q}) = a[\mathbf{p.A}][\mathbf{id.q}] : A\]
Then finally, we can holds 
\[\Gamma.\mathbf{Bool} \vdash \mathbf{if'}(a[\mathbf{id.true}], a[\mathbf{id.false}]) = a : A\]
This is proof of $\eta$-rule for $\mathbf{if'}$. 
\end{proof}

\newpage .

\end{document}